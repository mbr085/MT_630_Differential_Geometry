\section{Differential forms on \texorpdfstring{$\R^n$}  \phantom{s}}
\begin{frame}
  Recall that we have identified $T_p\R^n$ with
  % the vector space of
  point derivations $D \colon C^{\infty}_p(\R^n) \to \R$.

  $T_p \R^n$ has the canonical basis
  $\{
    \frac{\partial}{\partial x^1}\vert_p
    , \dots, 
    \frac{\partial}{\partial x^n}\vert_p
  \}$

  Let $U \subseteq \R^n$ be open. 

  A $C^{\infty}$ vector field on $U$ has the form $X = \sum_i a^i 
  \frac{\partial}{\partial x^i}$ where $a^i \in C^{\infty}(U)$.
  \begin{defn}
    A {\em covector field} on $U$ is a collection of covectors
    \begin{displaymath}
      \omega = \{ \omega_p \in (T_p U)^\vee \cond p \in U \}
    \end{displaymath}
  \end{defn}
  For each $p \in U$, there is a bilinear function
  \begin{displaymath}
    T_p \R^n \times C^{\infty}(U) \to \R, \quad (X_p, f) \mapsto X_p f
  \end{displaymath}
  \begin{defn}
    The {\em differential} of $f \in C^{\infty}(U)$ is the covector field
    \begin{displaymath}
      df = \{df_p \in (T_pU)^\vee \cond p \in U\}, \quad \text{
        where $df_p(X_p) = X_p f$
      }
    \end{displaymath}
  \end{defn}
\end{frame}
\begin{frame}
  The coordinate functions $x^1, \dots, x^n$ on $U$ give covector fields
  $dx^1, \dots, dx^n$.
  \begin{prop}
    For each $p \in U$, $dx^1_p, \dots, dx^n_p$
    is the dual basis associated to the basis 
    $\{
      \frac{\partial}{\partial x^1}\vert_p
      , \dots, 
      \frac{\partial}{\partial x^n}\vert_p
    \}$
  \end{prop}
  \begin{proof}
    \begin{displaymath}
    (dx^i_p)(
      \frac{\partial}{\partial x^j}\vert_p
    )
    =
      \frac{\partial}{\partial x^j}\vert_p
    (x^i)
    =
    \frac{\partial x^i}{\partial x^j}(p) 
    =
    \begin{cases}
      1 & \text{if $i = j$} \\
      0 & \text{if $i \ne j$}
    \end{cases}
    \end{displaymath}
  \end{proof}
  \begin{block}
    {Consequence}
    A covector field $\omega$ on $U$ can be written uniquely
    $\omega = \sum_{i=1}^n a_i dx^i$,
    where $a_i \colon U \to \R$ are functions for $i = 1, \dots, n$.
  \end{block}
  \begin{defn}
    $\omega$ is smooth if each $a_i$ is smooth. Write $\Omega^1(U)$ for the set
    of smooth covector fields on $U$.
  \end{defn}
\end{frame}
\begin{frame}
  \frametitle{Differential Forms}
  \begin{defn}
    Let $U \subseteq \R^n$ be an open subset.
    A {\em $k$-form} on $U$ is a collection of alternating
    $k$-linear functions
    \begin{displaymath}
      \omega = \{\omega_p \in A_k(T_pU) \cond p \in U\}
    \end{displaymath}
  \end{defn}
  \begin{example}
    The differential of $f \in C^{\infty}(U)$ is a $1$-form $df$ on $U$ 
    with $df_p$ defined by $df_pX_p = X_p f$.
    In particular, the coordinate functions $x^i \colon U \to \R$
    define $1$-forms $dx^1, \dots, dx^n$.
  \end{example}
  We have just checked that $dx^1_p, \dots, dx^n_p$ is the dual 
  basis for $(T_p\R^n)^\vee = A_i(T_p\R^n)$ with respect to the basis
  $\{
    \frac{\partial}{\partial x^1}\vert_p
    , \dots, 
    \frac{\partial}{\partial x^n}\vert_p
  \}$
  for $T_p\R^n$.

  We have seen that the $k$-covectors $dx_p^I$ for $I = i_1 < \dots < i_k$
  form a basis for $A_k(T_p U)$. Here 
  $dx^I = dx^{i_1} \wedge \dots \wedge dx^{i_n}$.
  \begin{defn}
    A $k$-form $\omega = \sum_I a_I dx^I$ is smooth if $a_I \colon U \to \R$
    is soooth for each $I$.
  \end{defn}
\end{frame}
\begin{frame}
  \begin{defn}
    $\Omega^k(U)$ is the real vector space of smooth $k$-forms on $U$.
  \end{defn}
  By convention $\Omega^0(U) = C^{\infty}(U)$. Note that $\Omega^k(U) = 0$
  for $k > n$. (Since $A_k(T_pU) = 0$ for $k > n$.)
  \begin{defn}
    The wedge product
    \begin{displaymath}
      \wedge \colon \Omega^k(U) \times \Omega^l(U) \to \Omega^{k+l}(U),
      \quad (\omega, \tau) \mapsto \omega \wedge \tau
    \end{displaymath}
    is defined by $(\omega \wedge \tau)_p = \omega_p \wedge \tau_p \in
    A_{k+l}(T_pU)$.
  \end{defn}
  Note that $\omega = \sum a_Idx^I$ and 
  $\tau = \sum_J b_J dx^J$
  gives 
  \begin{displaymath}
    \omega \wedge \tau =
    \sum_{I, J} (a_I b_J) dx^I dx^J 
    =
    \sum_{I, J \text{disjoint}} (a_I b_J) dx^I dx^J 
  \end{displaymath}
  For $I$ and $J$ disjoint, we can find a multiindex $K$ so that 
  $dx^K = \pm dx^I \wedge dx^J$. This shows that $\omega \wedge \tau$ is
  smooth.
\end{frame}
\begin{frame}
  \begin{prop}
    The wedge product makes 
    \begin{displaymath}
      \Omega^*(U) = \{\Omega^k(U) \cond k \ge 0 \}
    \end{displaymath}
    an anticommutative graded $\R$-algebra with unit $1 \in \Omega^0(U) =
    C^{\infty}(U).$
  \end{prop}
\end{frame}
\begin{frame}
  \frametitle{Exterior derivative}
  Let $U$ be an open subset of $\R^n$.
  \begin{block}
    {Goal}
    Construct the exterior derivative $d \colon \Omega^k(U) \to 
    \Omega^{k+1}(U)$.
  \end{block}
  The differential of a function defines an $\R$-linear function
  \begin{displaymath}
    d \colon C^{\infty}(U) = \Omega^0(U) \to \Omega^1(U), \quad f \mapsto df
  \end{displaymath}
  with $df_p X_p = X_p f$.
\end{frame}
\begin{frame}
  \begin{prop}
    If $f \in C^{\infty}(U)$, then $df = \sum_{i=1}^n 
    \frac{\partial f}{\partial x^i} dx^i$.
  \end{prop}
  \begin{proof}
    Write $f = \sum_i a_i dx^i$ for $a_i \colon U \to \R$, $i=1, \dots, n$.
    By construction
    \begin{displaymath}
      df_p(\frac{\partial}{\partial x^i}\vert_p) =
      \frac{\partial}{\partial x^i}\vert_p(f) =
      \frac{\partial f}{\partial x^i}(p)
    \end{displaymath}
    and
    \begin{displaymath}
      (\sum_{j=1}^n a_j(p) dx^j_p)(\frac{\partial}{\partial x^i}\vert_p)
      = a_i(p),
    \end{displaymath}
    so $a_i = \frac{\partial f}{\partial x_i}$.
  \end{proof}
  This shows that $df$ is a smooth $1$-form and that $d$ is $\R$-linear.
\end{frame}
\begin{frame}
  \begin{defn}
    For $k \ge 1$, the exterior derivative $d \colon \Omega^k(U) 
    \to \Omega^{k+1}(U)$ is the $\R$-linear function defined as follows:
    \begin{center}
      If $\omega = \sum_I a_I dx^I$, then $d\omega = \sum_I (da_I) \wedge dx^I$.
    \end{center}
  \end{defn}
  This expands to
  $
    d \omega = \sum_I \sum_i 
    \frac{\partial a_I}{\partial x^i} dx^i \wedge dx^I
    $
  \begin{example}
    $\omega = fdx + gdy$ on an open set $U \subseteq \R^2$.
    Then
    \begin{align*}
      d\omega &= df \wedge dx + dg \wedge dy \\
      &= 
      (
      \frac{\partial f}{\partial x} \wedge dx
      +
      \frac{\partial f}{\partial y} \wedge dy
      )
      \wedge dx
      +
      (
      \frac{\partial g}{\partial x} \wedge dx
      +
      \frac{\partial g}{\partial y} \wedge dy
      )
      \wedge dy \\
      &=
      \frac{\partial f}{\partial y} \wedge dy
      \wedge dx
      +
      \frac{\partial g}{\partial x} \wedge dx
      \wedge dy \\
      &=
      (
      \frac{\partial g}{\partial x}
      -
      \frac{\partial f}{\partial y}
      )
      dx \wedge dy
    \end{align*}
  \end{example}
\end{frame}
\begin{frame}
  \begin{example}
    $\omega = \frac{1}{x^2 + y^2}(-ydx + xdy)$ on $U = \R^2 \setminus \{0\}$.
    Notice that
    $\omega = fdx + gdy$, where $f(x,y) =
    \frac{-y}{x^2 + y^2}$ and
    $g(x, y) = 
    \frac{x}{x^2 + y^2}$.
    Using the rule for derivation of fractions, we get
    $\frac{\partial f}{\partial y} = \frac{-(x^2 + y^2) + 2y^2}{(x^2 +
    y^2)^2}$.
    Similarly 
    $\frac{\partial g}{\partial x} = \frac{(x^2 + y^2) - 2x^2}{(x^2 +
    y^2)^2}$.
    % 

    % In order to calculate $d \omega$, we decompose $\omega = \omega^1 +
    % \omega^2$,
    % where $\omega^1 = \frac{1}{x^2 + y^2}(-ydx) = a^1(x,y) dx$
    % and $\omega^2 = \frac{1}{x^2 + y^2}xdy$.
    % Since $dx \wedge dx = 0$, we have $d\omega^1 = 
    % \frac{\partial a^1}{\partial y}(x, y) dy \wedge dx 
    % =
    % -\frac{\partial a^1}{\partial y}(x, y) dx \wedge dy$.
    % Using the rule for derivation of fractions, we get
    % $\frac{\partial a^1}{\partial y} = \frac{-(x^2 + y^2) + 2y^2}{(x^2 +
    % y^2)^2}$.
    % Similarly 
    % $\frac{\partial a^2}{\partial x} = \frac{(x^2 + y^2) - 2x^2}{(x^2 +
    % y^2)^2}$.
    In conclusion
    \begin{displaymath}
      d\omega = \frac{(x^2 + y^2) - 2x^2 +(x^2 + y^2) - 2y^2}{(x^2 + y^2)^2}
      dx \wedge dy
      = 0
    \end{displaymath}
  \end{example}
\end{frame}
\begin{frame}
  \begin{prop}
    Properties of $d \colon \Omega^k(U) \to \Omega^{k+1}(U)$.
    \begin{enumerate}[(i)]
      \item $d(\omega \wedge \tau) = (d\omega) \wedge \tau + (-1)^{\deg \omega}
        \omega \wedge (d \tau)$
      \item $d \circ d = 0 \colon \Omega^k(U) \to \Omega^{k+2}(U)$
      \item If $X$ is a smooth vector field on $U$ and $f \in \Omega^0(U) =
        C^{\infty}(U)$, then $df_p(X_p) = X_p(f)$ for all $p \in U$.
    \end{enumerate}
  \end{prop}
  \begin{proof}
    (i): By linearity, it suffices to consider $\omega = fdx^I \in \Omega^k(U)$ and 
    $\tau = gdx^J \in \Omega^l(U)$.
    Then
    \begin{align*}
      d(\omega \wedge \tau) &=
      d(fg dx^I \wedge dx^J) \\
      &=
      \sum_i \frac{\partial fg}{\partial x^i} 
      dx^i \wedge
      dx^I \wedge dx^J \\
      &=
      \sum_i \frac{\partial f}{\partial x^i}g 
      dx^i \wedge
      dx^I \wedge  dx^J +
      \sum_i f \frac{\partial g}{\partial x^i} 
      dx^i \wedge
      dx^I \wedge dx^J 
    \end{align*}
  \end{proof}
\end{frame}

\begin{frame}
  \begin{proof}[Proof (cont.)]
    (i): By linearity, it suffices to consider $\omega = fdx^I \in \Omega^k(U)$ and 
    $\tau = gdx^J \in \Omega^l(U)$.
    Then
    \begin{align*}
      d(\omega \wedge \tau) &=
      d(fg dx^I \wedge dx^J) \\
      &=
      \sum_i \frac{\partial fg}{\partial x^i} 
      dx^i \wedge
      dx^I \wedge dx^J \\
      &=
      \sum_i \frac{\partial f}{\partial x^i}g 
      dx^i \wedge
      dx^I \wedge  dx^J +
      \sum_i f \frac{\partial g}{\partial x^i} 
      dx^i \wedge
      dx^I \wedge dx^J 
    \end{align*}
    Now, $dx^i \wedge dx^I \wedge dx^J = (-1)^k dx^I \wedge dx^i \wedge dx^J$, 
    so
    \begin{align*}
      d(\omega \wedge \tau) &=
      \sum_i \frac{\partial f}{\partial x^i}g 
      dx^i \wedge
      dx^I \wedge  dx^J +
      (-1)^k
      \sum_i f \frac{\partial g}{\partial x^i} 
      dx^I
      \wedge dx^i
      \wedge dx^J  \\
      &= (d\omega) \wedge \tau + (-1)^k \omega \wedge (d\tau)
    \end{align*}
  \end{proof}
\end{frame}

\begin{frame}
  \begin{proof}[Proof (cont.)]
    (ii): By linearity, it suffices to
    show that $d(d(\omega)) = 0$ for $\omega = fdx^I$.
    Direct calculation gives
    \begin{displaymath}
      d(d(fdx^I)) = d(\sum_i 
      \frac{\partial f}{\partial x^i} dx^i \wedge dx^I)
      =
      \sum_{i, j}
      \frac{\partial^2 f}{\partial x^j \partial x^i} dx^j \wedge dx^i \wedge dx^I)
    \end{displaymath}
    Here, if $i = j$, then $dx^j \wedge dx^i = 0$. If $i < j$, then
    $dx^j \wedge dx^i = - dx^i \wedge dx^j$, and 
    $
    \frac{\partial^2 f}{\partial x^j \partial x^i} = 
    \frac{\partial^2 f}{\partial x^i \partial x^j}
    $,
    so these terms cancel. This shows that $d^2(fdx^I) = d(d(fdx^I)) = 0$.

    \vspace{1cm}

    (iii): The formula $df_p(X_p) = X_p(f)$ is actually the definition of $df$.
  \end{proof}
\end{frame}
\begin{frame}
  \begin{prop}[Uniqueness of exterior derivative]
    Let $D \colon \Omega^k(U) \to \Omega^{k+1}(U)$ be a collection
    of $\R$-linear functions for $k \ge 0$ and suppose that
    \begin{enumerate}[(i)]
      \item $D(\omega \wedge \tau)= D(\omega) \wedge \tau + (-1)^{\deg \omega}
        \omega \wedge D(\tau)$
      \item $D \circ D = 0$
      \item For $X$ a smooth vector field and $f \in \Omega^0(U) = C^{\infty}(U)$,
        we have $(Df)_p(X_p) = X_p(f)$ for all $p \in U$.
    \end{enumerate}
    Then $D = d$ is the exterior derivative.
  \end{prop}
  Remark that we have proved that $d$ satisfies (i), (ii) and (iii).
\end{frame}

\begin{frame}
  \begin{proof}
    By linearity it suffices to show that $D \omega = d \omega$
    for $\omega = fdx^I$.

    By (i) $D\omega = D(f) \wedge dx^I + fD(dx^I)$.

    By (ii) $D f = df$, so
    \begin{displaymath}
      D(f) \wedge dx^I = df \wedge dx^I = d(f dx^I) = d\omega
    \end{displaymath}
    It remains to show that $D(dx^I) = 0$.
    \begin{description}
      \item[$k=1$] $D(dx^i) = DDx^i = 0$ by (ii)
      \item[$k=2$] $D(dx^i \wedge x^j) = D(dx^i)\wedge dx^j
        - dx^i \wedge D(dx^j) = 0 - 0 = 0$
      \item[$k=3$] Let $\omega = dx^i \wedge dx^I$ with $I = (i_1 < i_2)$.
        Then $D\omega = D(dx^i \wedge dx^I) = D(dx^i) \wedge dx^I
        - dx^i \wedge D(x^I) = 0 - 0 = 0$
    \end{description}
    General case by induction.
  \end{proof}
\end{frame}
\begin{frame}
  \frametitle{Applications to vector calculus}
  Let $U$ be an open subset of $\R^3$.
  A smooth vector field $X$ on $U$ can be written
  \begin{displaymath}
    X =
    P \frac{\partial}{\partial x}
    +
    Q \frac{\partial}{\partial y}
    +
    R \frac{\partial}{\partial z}
  \end{displaymath}
  We can identify $X$ with the vector function 
  \begin{displaymath}
    \begin{bmatrix}
      P \\ Q \\ R
    \end{bmatrix}
    \colon U \to \R^3
  \end{displaymath}
  Let $\vfields(U)$ be the vector space of smooth vector fields on $U$.
  There are linear maps
  \begin{displaymath}
    \operatorname{grad} \colon C^{\infty}(U) \to \vfields(U), \quad f \mapsto 
    \begin{bmatrix}
      {\partial f} / {\partial x} \\
      {\partial f} / {\partial y} \\
      {\partial f} / {\partial z} \\
    \end{bmatrix}
  \end{displaymath}
\end{frame}
\begin{frame}[fragile]
  \begin{displaymath}
    \operatorname{curl} \colon \vfields(U) \to \vfields(U), \quad
    \begin{bmatrix}
      P \\ Q \\ R
    \end{bmatrix}
    \mapsto
    \begin{bmatrix}
      {\partial} / {\partial x} \\
      {\partial} / {\partial y} \\
      {\partial} / {\partial z} \\
    \end{bmatrix}
    \times
    \begin{bmatrix}
      P \\ Q \\ R
    \end{bmatrix}
    =
    \begin{bmatrix}
      \partial R / \partial y - \partial Q / \partial z \\
      -(\partial R / \partial x - \partial P / \partial z) \\
      \partial Q / \partial x - \partial P / \partial y
    \end{bmatrix}
  \end{displaymath}
  \begin{displaymath}
    \operatorname{div} \colon \vfields(U) \to C^{\infty}(U), \quad
    \begin{bmatrix}
      P \\ Q \\ R
    \end{bmatrix}
    \mapsto 
    \partial P / \partial x
    +
    \partial Q / \partial y
    +
    \partial R / \partial z
  \end{displaymath}
  \begin{prop}
    There is a commutative diagram
    \begin{displaymath}
      \begin{tikzcd}[row sep=huge]
        \Omega^0(U) \arrow[r, "d"] \arrow[d, <->, "="] &
        \Omega^1(U)
        \arrow[r, "d"] \arrow[d, <->, "\cong"] &
        \Omega^2(U)
        \arrow[r, "d"] \arrow[d, <->, "\cong"] &
        \Omega^3(U)\arrow[d, <->, "\cong"] 
        \\
        C^{\infty}(U) 
        \arrow[r, "\operatorname{grad}"]
        &
        \vfields(U)
        \arrow[r, "\operatorname{curl}"]
        &
        \vfields(U)
        \arrow[r, "\operatorname{div}"]
        &
        C^{\infty}(U) 
      \end{tikzcd}
    \end{displaymath}
  \end{prop}
\end{frame}
\begin{frame}
  The vertical isomorphisms are:
  \begin{displaymath}
    \Omega^1(U) \leftrightarrow \vfields(U), \quad
    Pdx + Qdy + Rdz \leftrightarrow
    \begin{bmatrix}
      P \\ Q \\ R
    \end{bmatrix}
  \end{displaymath}
  \begin{displaymath}
    \Omega^2(U) \leftrightarrow \vfields(U), \quad
    P dy \wedge dz + Q dz \wedge dx + R dx \wedge dy 
    \leftrightarrow 
    \begin{bmatrix}
      P \\ Q \\ R
    \end{bmatrix}
  \end{displaymath}
  \begin{displaymath}
    \Omega^3(U) \leftrightarrow C^{\infty}(U), \quad
    f dx \wedge dy \wedge dz \leftrightarrow f
  \end{displaymath}
  We check the middle square in the diagram:
  \begin{align*}
    &d(Pdx + Qdy + Rdz) 
    \\
    &=
    \phantom{+}
    \frac{\partial P}{\partial dy} dy \wedge dx +
    \frac{\partial P}{\partial dx} dz \wedge dx
    \\
    &\phantom{=}
    +
    \frac{\partial Q}{\partial dx} dx \wedge dy +
    \frac{\partial Q}{\partial dz} dz \wedge dy
    \\
    &\phantom{=}
    +
    \frac{\partial R}{\partial dx} dx \wedge dz +
    \frac{\partial R}{\partial dy} dy \wedge dz 
    \\
    &=
      (\partial R / \partial y - \partial Q / \partial z) dy \wedge dz 
      -(\partial R / \partial x - \partial P / \partial z) dz \wedge dx
      + (\partial Q / \partial x - \partial P / \partial y) dx \wedge dy
  \end{align*}
\end{frame}
