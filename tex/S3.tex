\section{The Exterior Algebra of Multicovectors}
\begin{frame}
  \frametitle{Multilinear Algebra}
  We introduce the Exterior Algebra. This is where tensors in tensor calculus
  live.

  The idea is that when linearizing we obatin tangent spaces, and 
  we work with these instead of directly with mainfolds. This leads 
  into multilinear algebra.
\end{frame}
\begin{frame}
  \frametitle{Dual Space}
  \begin{block}
    {Notation}
    Unless otherwise stated, $V$ will be a real vector
    space of dimension $n$, and $e_1, \dots, e_n$ will be
    a basis for $V$.
  \end{block}
  \begin{definition}
    The {\em dual} vector space of $V$ is
    \begin{displaymath}
      V\dual = \Hom(V, \R),
    \end{displaymath}
    the vector space of linear maps form $V$ to $\R$.
    Elements of $V\dual$ are called {\em covectors}.
  \end{definition}
  \begin{exercise}
    Explain how sum and scalar multiplication makes $V\dual$ a vector space.
  \end{exercise}
\end{frame}

\begin{frame}
  The covectors $\alpha^1, \dots, \alpha^n$ are defined by $\alpha^i(v) = v^i$, 
  where $v^i$ is the $i$-th coordinate of $v = v^1 e_1 + \dots + v^n e_n$.
  \begin{block}
    {Observation}
    \begin{displaymath}
      \alpha^i(e_j) = 
      \begin{cases}
        1 & \text{if $i = j$} \\
        0 & \text{otherwise}
      \end{cases}
    \end{displaymath}
  \end{block}
    \begin{prop}
      The covectors $\alpha^1, \dots, \alpha^n$ form a basis for $V\dual$.
    \end{prop}
    \begin{proof}[Proof (Linearly independent:)]
      Assume $c_1 \alpha^1 + \dots + c_n \alpha^n = 0$
      for $c_1, \dots, c_n \in \R$. Evaluate at $e_i$ to get
      \begin{displaymath}
        0 = (c_1 \alpha^1 + \dots + c_n \alpha^n )(e_i) = c_i \alpha^i(e_i) =
        c_i.
      \end{displaymath}
    \end{proof}
\end{frame}

\begin{frame}
    \begin{prop}
      The covectors $\alpha^1, \dots, \alpha^n$ form a basis for $V\dual$.
    \end{prop}
    \begin{proof}[Proof (the $\alpha^i$ generate: $V\dual$)]
      Let $f \in V\dual$ and $v = v^1e_1 + \dots + v^n e_n \in V$.
      Then 
      \begin{displaymath}
        f(v) = v^1f(e_1) + \dots + v^nf(e_n) =
        f(e_1) \alpha^1(v) + \dots
        f(e_n) \alpha^n(v)
      \end{displaymath}
      This reads $f = f(e_1) \alpha^1 + \dots + f(e_n) \alpha^n.$
    \end{proof}
    \begin{block}
      {Consequence}
      $V\dual$ has the same dimension as $V$.
    \end{block}
    \begin{remark}
      In linear algebra vectors are column vectors and covectors are row
      vectors. Multiplication by a fixed row vector gives a linear map
      from column vectors to the real numbers.
    \end{remark}
\end{frame}
\begin{frame}
  Let $V^k = V \times \dots \times V$ be the product of $k$ copies of $V$.
  \begin{defn}
    A function $f \colon V^k \to \R$ is {\em $k$-linear} if it is
    linear in each or the $k$ variables:
    \begin{displaymath}
      f(v_1, \dots,  x + y, \dots, v_n) =
      f(v_1, \dots,  x, \dots, v_n) +
      f(v_1, \dots,  y, \dots, v_n)
    \end{displaymath}
    and
    \begin{displaymath}
      f(v_1, \dots,  cx, \dots, v_n) =
      cf(v_1, \dots,  x, \dots, v_n) 
    \end{displaymath}
  \end{defn}
  \begin{example}
    The scalar product on $\R^n$ is bilinear ($2$-linear),
    \begin{displaymath}
      \R^n \times \R^n \to \R^n, \quad (x, y) \mapsto x \cdot y = x^1 y^1 + \dots
      + x^n y^n
    \end{displaymath}
  \end{example}
  \begin{example}
    The determinant $\det: \R^n \times \dots \times \R^n \to \R$ is $n$-linear.
  \end{example}
\end{frame}
\begin{frame}
  Write $L_k(V)$ for the vector space of $k$-linear functions.
  \begin{exercise}
    Explain how sum and scalar multiplication makes $L_k(V)$ a vector space.
    Show that the dimension of $L_k(V)$ is $n^k$.
  \end{exercise}
  Convention $L_0(V) = \R$ and $L_1(V) = V\dual$.
\end{frame}
\begin{frame}
  \frametitle{Review of permutations}
  Let $A = \{1, \dots, k\}$. A {\em permutation} of $A$ is a bijective 
  function $\sigma \colon A \to A$.

  Let $S_k$ be the set of all permutations
  of $A$.

  Permutations can be composed $\tau \circ \sigma$

  This makes $S_k$ a group with unit element the identity function.

  A permutation $\sigma$ can be described by its matrix
  \begin{displaymath}
    \sigma = 
    \begin{bmatrix}
      1 & 2 & \dots & k \\
      \sigma(1) & \sigma(2) & \dots & \sigma(k)
    \end{bmatrix}
  \end{displaymath}
\end{frame}
\begin{frame}
  An {\em inversion} in $\sigma$ is a pair $(\sigma(i), \sigma(j))$ with
  $\sigma(i) > \sigma(j)$ and $i < j$.
  \begin{example}
  \begin{displaymath}
    \sigma = 
    \begin{bmatrix}
      1 & 2 & 3 & 4 & 5 \\
      3 & 4 & 5 & 2 & 1
    \end{bmatrix}
  \end{displaymath}
    Inversions in $\sigma$ are $(3, 1), (4, 1), (5, 1), (2,1), (3, 2), (4, 2)$
    and
    $(5, 2)$. There are $7$ inversions in $\sigma$.
  \end{example}
  \begin{defn}
    A permutation $\sigma$ is {\em even} if it han an even number of
    inversions. Otherwise it is odd.
    The sign of $\sigma$ is
    \begin{displaymath}
      \sgn(\sigma) = 
      \begin{cases}
        1 & \text{if $\sigma$ is even} \\
        -1 & \text{if $\sigma$ is odd}
      \end{cases}
    \end{displaymath}
  \end{defn}
  \begin{block}
    {Fact from algebra}
    $\sgn(\tau \circ \sigma) = \sgn(\tau) \cdot \sgn(\sigma)$
  \end{block}
\end{frame}
\begin{frame}
  \frametitle{Multilinear Functions}
  Let $f \colon V^k \to \R$ be a $k$-linear function.
  \begin{defn}
    $f$ is {\em symmetric} if
    $f(v_{\sigma(1)}, \dots, v_{\sigma(k)}) = f(v_1, \dots, v_k)$
    for every $\sigma \in S_k$.
    \begin{displaymath}
      S_k(V) = \{f \in L_k(V) \cond \text{$f$ is symmetric}\}
    \end{displaymath}
  \end{defn}
  \begin{defn}
    $f$ is {\em alternating} if
    $f(v_{\sigma(1)}, \dots, v_{\sigma(k)}) = \sgn(\sigma) f(v_1, \dots, v_k)$
    for every $\sigma \in S_k$.
    \begin{displaymath}
      A_k(V) = \{f \in L_k(V) \cond \text{$f$ is alternating}\}
    \end{displaymath}
  \end{defn}
  By convention $A_0(V) = S_0(V) = L_0(V) = \R$ and $A_1(V) = S_1(V) = L_1(V) =
  V\dual$
\end{frame}
\begin{frame}
  \begin{example}
    The scalar product on $\R^n$ is symmetric
    \begin{displaymath}
      \R^n \times \R^n \to \R^n, \quad (x, y) \mapsto x \cdot y = x^1 y^1 + \dots
      + x^n y^n
    \end{displaymath}
  \end{example}
  \begin{example}
    The determinant $\det: \R^n \times \dots \times \R^n \to \R$ is alternating
  \end{example}
  \begin{example}
    Given covectors $f \colon V \to \R$ and $g \colon V \to \R$, we can define 
    an alternating $2$-linear function $f \wedge g \colon V^2 \to \R$ by
    \begin{displaymath}
      (f \wedge g)(v, w) = f(v)g(w) - f(w) g(v)
    \end{displaymath}
  \end{example}
\end{frame}
\begin{frame}
  There is a function 
  \begin{displaymath}
    S_k \times L_k(V) \to L_k(V), \quad (\sigma, f) \mapsto \sigma f
  \end{displaymath}
  where 
  \begin{displaymath}
    (\sigma f)(v_1, \dots, v_k) = f(v_{\sigma(1)}, \dots, v_{\sigma(k)})
  \end{displaymath}
  \begin{block}
    {Observation}
    $f$ is symmetric if and only if $\sigma f = f$ for all $\sigma \in S_k$

    $f$ is alternating if and only if $\sigma f = \sgn(\sigma)f$ for all
    $\sigma \in S_k$
  \end{block}
  \begin{lemma}
    $(\tau \sigma) f = \tau(\sigma f)$ for all $\tau, \sigma \in S_k$ and $f
    \in L_k(V)$.
  \end{lemma}
  \begin{proof}
    Let $w_i = v_{\tau(i)}$. Then $w_{\sigma(i)} = v_{\tau(\sigma(i))} =
    v_{(\tau \sigma)(i)}$ and
    \begin{displaymath}
      \tau(\sigma f)(v_1, \dots, v_k) = (\sigma f)(w_1, \dots, w_k)
      = f(w_{\sigma(1)}, \dots, w_{\sigma(k)}) =
      f(v_{(\tau \sigma(1))}, \dots, v_{(\tau \sigma)(k)})
    \end{displaymath}
  \end{proof}
\end{frame}
\begin{frame}
  The above lemma states that $(\sigma, f) \mapsto \sigma f$ is an
  action of the group $S_k$ on the set $L_k(V)$.
  \begin{definition}
    For $f \in L_k(V)$ we define
    \begin{displaymath}
      S(f) = \sum_{\sigma \in S_k} \sigma f
      \quad \text{and} \quad
      A(f) = \sum_{\sigma \in S_k} \sgn(\sigma)\sigma f
    \end{displaymath}
  \end{definition}
  \begin{prop}
    $S(f)$ is symmetric and $A(f)$ is alternating.
  \end{prop}
  \begin{proof}[Proof that $A(f)$ is alternating]
    \begin{align*}
      \tau(Af) &= \sum_{\sigma} \tau(\sgn(\sigma) \sigma f) =
      \sum_{\sigma} \sgn(\sigma) \tau(\sigma f)\\
      &= \sum_\sigma \sgn(\sigma) (\tau \sigma) f =
      \sgn(\tau) \sum_{\sigma} \sgn(\tau \sigma) (\tau \sigma) f = \sgn(\tau) Af
    \end{align*}
  \end{proof}
\end{frame}
\begin{frame}
  \begin{lemma}
    If $f \in A_k(V)$, then $A(f) = (k!)f$.
  \end{lemma}
  \begin{proof}
    \begin{displaymath}
      A(f) = \sum_{\sigma} (\sgn \sigma) \sigma f =
      \sum_{\sigma} (\sgn \sigma) (\sgn \sigma) f = (k!) f
    \end{displaymath}
  \end{proof}
\end{frame}
\begin{frame}
  \frametitle{The Wedge Product}
  First define the tensor product of multilinear functions
  \begin{displaymath}
    \otimes \colon L_k(V) \times L_l(V) \to L_{k+l}(V),
    \quad (f, g) \mapsto f \otimes g
  \end{displaymath}
  where
  \begin{displaymath}
    (f \otimes g)(v_1, \dots, v_k, v_{k+1}, \dots, v_{k+l}) = 
    f(v_1, \dots, v_k) \cdot 
    g(v_{k+1}, \dots, v_{k+l})
  \end{displaymath}
  If $k = 0$, then $f = c \in \R$ and $f \otimes g = c \otimes g = cg$.
  \begin{block}
    {Observation}
    $\otimes$ is bilinear and associative, $(f \otimes g) \otimes h =
    f \otimes (g \otimes h)$.
  \end{block}
  \begin{definition}
    The {\em wedge product} is the bilinear function
    \begin{displaymath}
      \wedge \colon A_k(V) \times A_l(V) \to A_{k+l}(V), \quad
      (f, g) \mapsto f \wedge g = \frac{1}{k! l!}A(f \otimes g)
    \end{displaymath}
  \end{definition}
\end{frame}
\begin{frame}
  \begin{definition}
    The {\em wedge product} is the bilinear function
    \begin{displaymath}
      \wedge \colon A_k(V) \times A_l(V) \to A_{k+l}(V), \quad
      (f, g) \mapsto f \wedge g = \frac{1}{k! l!}A(f \otimes g)
    \end{displaymath}
  \end{definition}
  Note that we know that $f \wedge g$ is alternating.
  Explicitly:
  \begin{displaymath}
    (f \wedge g)(v_1, \dots, v_{k+l}) =
    \frac{1}{k!l!} \sum_{\sigma \in S_{k+l}}
    f(v_{\sigma(1)}, \dots, v_{\sigma(k)})
    g(v_{\sigma(k+1)}, \dots, v_{\sigma(k+l)})
  \end{displaymath}
  For $k = 0$, $f \in A_0V = \R$ is a constant $c$, and
  \begin{displaymath}
    (c \wedge g)(v_1, \dots, v_l) = \frac{1}{l!} 
    \sum_{\sigma \in S_l} cg(v_{\sigma(1)}, \dots, v_{\sigma(l)}) =
    cg(v_1, \dots, v_l)
  \end{displaymath}
  so $c \wedge g = cg$ for $c \in \R$ and $g \in A_l(V)$.
  Note that the coefficient $\frac{1}{l!}$ compensates for repetition. 
  So does the general coefficient $\frac{1}{k!l!}$.
\end{frame}
\begin{frame}
  \begin{example}
    $f \in A_2(V)$, $g \in A_1(V)$. Then
    \begin{align*}
      A(f \otimes f) (v_1, v_2, v_3) &=
      \phantom{-}
      f(v_1,v_2)g(v_3) - f(v_1, v_3)g(v_2) + f(v_2, v_3) g(v_1) \\
      &\phantom{=}-
      f(v_2,v_1)g(v_3) - f(v_3, v_1)g(v_2) + f(v_3, v_2) g(v_1) \\
      &=2(
      f(v_1,v_2)g(v_3) - f(v_1, v_3)g(v_2) + f(v_2, v_3) g(v_1) )
    \end{align*}
  \end{example}
  \begin{definition}
    $\sigma \in S_{k+l}$ is a $(k,l)$-shuffle if both
    \begin{displaymath}
      \sigma(1) < \dots < \sigma(k)
      \quad \text{ and } \quad
      \sigma(k+1) < \dots < \sigma(k+l)
    \end{displaymath}
  \end{definition}
  \begin{prop}
    \begin{displaymath}
      (f \wedge g)(v_1, \dots, v_{k+l}) = \sum_{(k,l)-\text{shuffle } \sigma}
      f(v_{\sigma(1)}, \dots, v_{\sigma(k)})
      g(v_{\sigma(k+1)}, \dots, v_{\sigma(k+l)})
    \end{displaymath}
  \end{prop}
\end{frame}
\begin{frame}
  The wedge product is anticommutative:
  \begin{displaymath}
    f \wedge g = (-1)^{kl} g \wedge f \quad \text{for $f \in A_k(V)$
    and $g \in A_l(V)$}
  \end{displaymath}
  \begin{proof}
    Let 
    \begin{displaymath}
      \tau = 
      \begin{bmatrix}
        1 & \dots & l & l+1 & \dots l+k \\
        k+1 & \dots & k+ l & 1 & \dots k
      \end{bmatrix}
    \end{displaymath}
    Then
    \begin{displaymath}
      \sigma(1) = \sigma \tau(l+1), \dots, \sigma(k) = \sigma \tau(l+k)
    \end{displaymath}
    and
    \begin{displaymath}
      \sigma(k+1) = \sigma \tau(1), \dots, \sigma(k+l) = \sigma \tau(l)
    \end{displaymath}
  \end{proof}
\end{frame}

\begin{frame}
  \begin{proof}[Proof (cont.)]
    For $v_1, \dots, v_{k+l} \in V$ we have
    \begin{align*}
      \phantom{=}&
      A(f \otimes g)(v_1, \dots, v_{k+l}) \\
      =&
      \sum_{\sigma} \sgn(\sigma)
      f(v_{\sigma(1)}, \dots, v_{\sigma(k)})
      g(v_{\sigma(k+1)}, \dots, g_{\sigma(k+l)}) \\
      =&
      \sum_{\sigma} \sgn(\sigma)
      f(v_{\sigma\tau(l+1)}, \dots, v_{\sigma\tau(l+k)})
      g(v_{\sigma\tau(1)}, \dots, g_{\sigma\tau(l)})
      \\
      =&
      \sum_{\sigma} \sgn(\sigma)
      g(v_{\sigma\tau(1)}, \dots, g_{\sigma\tau(l)})
      f(v_{\sigma\tau(l+1)}, \dots, v_{\sigma\tau(l+k)})
      \\
      =&
      \sgn(\tau)\sum_{\sigma} \sgn(\sigma\tau)
      g(v_{\sigma\tau(1)}, \dots, g_{\sigma\tau(l)}) 
      f(v_{\sigma\tau(l+1)}, \dots, v_{\sigma\tau(l+k)})
      \\
      =&
      \sgn(\tau) A(g \otimes f)(v_1, \dots, v_{k+l})
    \end{align*}
  \end{proof}
\end{frame}
\begin{frame}
  \begin{proof}[Proof (cont.)]
    We conclude that
    \begin{displaymath}
      A(f \otimes g)(v_1, \dots, v_{k+l}) 
      =
      \sgn(\tau) A(g \otimes f)(v_1, \dots, v_{k+l})
    \end{displaymath}
    Dividing by $k!l!$ on both sides gives $f \wedge g = \sgn(\tau) g \wedge
    f$.
    Now, $\sgn(\tau) = (-1)^{kl}$ because $\tau$ has $kl$ inversions.
  \end{proof}
  \begin{cor}
    If $k$ is odd and $f \in A_k(V)$, then $f \wedge f = 0$.
  \end{cor}
  \begin{proof}
    \begin{displaymath}
      f \wedge f = (-1)^{k^2} f \wedge f = - f \wedge f
      \Rightarrow f \wedge f = 0
    \end{displaymath}
  \end{proof}
\end{frame}
\begin{frame}
  \begin{prop}
    The wedge product is associative:
    \begin{displaymath}
      (f \wedge g) \wedge h = f \wedge (g \wedge h)
    \end{displaymath}
    for $f \in A_k(V)$, $g \in A_l(V)$ and $h \in A_m(V)$.
  \end{prop}
\end{frame}
\begin{frame}
  \begin{lem}
    For $f \in A_k(V)$ and $g \in A_l(V)$, we have
    \begin{enumerate}[(i)]
      \item $A(A(f) \otimes g) = k!A(f \otimes g)$
      \item $A(f \otimes A(g)) = k!A(f \otimes g)$
    \end{enumerate}
  \end{lem}
  \begin{proof}[Proof of (i)]
    First note that
    \begin{displaymath}
      A(A(f) \otimes g) = A(\sum_{\tau \in S_k} (\sgn(\tau) \tau f) \otimes f)
      = \sum_{\tau \in S_k} A((\sgn(\tau) \tau f) \otimes g)
    \end{displaymath}
    Given $\tau \in S_k \subseteq S_{k+l}$ we have $(\tau f) \otimes g = \tau(f
    \otimes g)$.
    Therefore
    \begin{align*}
      A((\sgn(\tau) \tau f) \otimes g) &=
      \sum_{\sigma \in S_{k+l}} \sgn(\sigma)\sigma((\sgn(\tau) \tau f) \otimes g)
      =
      \sum_{\sigma \in S_{k+l}} \sgn(\sigma) \sgn(\tau)
      \sigma(\tau(f \otimes g)) 
      \\
      &=
      \sum_{\sigma \in S_{k+l}} \sgn(\sigma \tau)
      (\sigma \tau)(f \otimes g))
      =
      \sum_{\sigma \in S_{k+l}} \sgn(\sigma)
      (\sigma)(f \otimes g)) = A(f \otimes g)
    \end{align*}
  \end{proof}
\end{frame}

\begin{frame}
  \begin{prop}
    The wedge product is associative:
    \begin{displaymath}
      (f \wedge g) \wedge h = f \wedge (g \wedge h)
    \end{displaymath}
    for $f \in A_k(V)$, $g \in A_l(V)$ and $h \in A_m(V)$.
  \end{prop}
  \begin{proof}
    This is a direct calculation. Part (i) of the lemma gives:
    \begin{align*}
      (f \wedge g) \wedge h &= \frac{1}{(k+l)!m!} A((f \wedge g) \wedge h) \\
      &=
      \frac{1}{(k+l)!m!k!l!} A(A(f \otimes g) \otimes h)
      \\
      &=
      \frac{(k+l)!}{(k+l)!m!k!l!} A((f \otimes g) \otimes h)
      =
      \frac{1}{k!l!m!} A((f \otimes g) \otimes h)
    \end{align*}
    Similarly, part (ii) of the lemma gives $f \wedge (g \wedge h) = 
    \frac{1}{k!l!m!} A(f \otimes (g \otimes h))$.
  \end{proof}
\end{frame}
\begin{frame}
  Note that we have proved that $f \wedge g \wedge h = \frac{1}{k!l!m!}
  A(f \otimes g \otimes h)$ for $f \in A_k(V)$, $g \in A_l(V)$ and $h \in
  A_m(V)$.

  Let $k \in A_n(V)$. Then
  \begin{align*}
    \phantom{=}&f \wedge g \wedge h \wedge k = (f \wedge g \wedge h) \wedge k \\
    =&
    \frac{1}{(k+l+m)!n!} A((f \wedge g \wedge h) \otimes k)
    = \frac{1}{(k+l+m)! n!} A(\frac{1}{k!l!m!} A(f \otimes g \otimes h) \otimes
    k) \\
    =& 
    \frac{1}{k!l!m!n!} A(\frac{1}{(k+l+m)!n!} A(f \otimes g \otimes h) \otimes k)
    =
    \frac{1}{k!l!m!n!} A(f \otimes g \otimes h \otimes k)
  \end{align*}
  By induction, if $f_1 \in A_{k_1}(V), \dots, f_r \in A_{k_r}(V)$, then
  \begin{displaymath}
    f_1 \wedge \dots \wedge f_r =
    \frac{1}{k_1! \dots k_r!}A(f_1 \otimes \dots \otimes f_r)
  \end{displaymath}
\end{frame}

\begin{frame}
  \begin{example}
    Let $\alpha^1, \dots, \alpha^k \in A_1(V)$.
    Then $\alpha^1 \wedge \dots \wedge \alpha^k \in A_k(V)$ is given by
    \begin{displaymath}
      \alpha^1 \wedge \dots \wedge \alpha^k  = 
      A(\alpha^1 \otimes \dots \otimes \alpha^k )
      =
      \sum_{\sigma_i \in S_k} \sgn(\sigma) \sigma(
      \alpha^1 \otimes \dots \otimes \alpha^k 
      )
    \end{displaymath}
    Given $v_1, \dots, v_k \in V$, we get
    \begin{align*}
      \alpha^1 \wedge \dots \wedge \alpha^k
      (
      v_1,\dots, v_k
      )
      &=
      \sum_{\sigma_i \in S_k} \sgn(\sigma) \sigma(
      \alpha^1 \otimes \dots \otimes \alpha^k 
      )
      (
      v_1,\dots, v_k
      ) 
      \\
      &=
      \sum_{\sigma_i \in S_k} \sgn(\sigma) 
      \alpha^1 \otimes \dots \otimes \alpha^k 
      (
      v_{\sigma(1)},\dots, v_{\sigma(k)}
      ) 
      \\
      &=
      \sum_{\sigma_i \in S_k} \sgn(\sigma) 
      \alpha^1(v_{\sigma(1)})
      \otimes \dots \otimes \alpha^k (v_{\sigma(k)}) \\
    \end{align*}
    This is the determinant $\det(\alpha^i(v_j))$ of the matrix with $(i,j)$-entry
    $\alpha^i(v_j)$
  \end{example}
\end{frame}
\begin{frame}
  \begin{defn}
    A {\em graded $\R$-algebra} $A = \{A(k) \cond k \ge 0\}$
    is a collection of $\R$-vector spaces $A(k)$, together with
    \begin{enumerate}
      \item unit $1 \in A(0)$
      \item bilinear multiplication $A(k) \times A(l) \to A(k+l)$,
        $(a,b) \mapsto a \cdot b$
    \end{enumerate}
    such that the multiplication is
    \begin{enumerate}[(i)]
      \item unital: $a\cdot 1 = a$, $1 \cdot a = a$
      \item associative: $(a \cdot b) \cdot c = a \cdot (b \cdot c)$.
    \end{enumerate}
    A is {\em anticommutative} if $ab = (-1)^{kl}$
    for $a \in A(k)$ and $b \in A(l)$.
  \end{defn}
  \begin{block}
    {Summary}
    $A_*(V) = \{A_k(V) \cond k \ge 0\}$ is an anticommutative 
    graded algebra when $a \cdot b = a \wedge b$.
    This is the {\em exterior algebra} of $V$.
  \end{block}
\end{frame}
\begin{frame}
  \frametitle{A basis for $A_k(V)$}
  Let $e_1, \dots, e_n$ be a basis for $V$.
  We write $\alpha^1, \dots, \alpha^n$
  for the dual basis for $A_1(V) = V^\vee$ determined by
  $\alpha^i(e_j) = \delta^i_j$.
  Given $I = (i_1, \dots, i_k)$ with each $i_s \in \{1,\dots, n\}$
  we write 
  \begin{displaymath}
    \alpha^I = \alpha^{i_1} \wedge \dots \wedge \alpha^{i_k}
    \quad \text{and} \quad
    e_I = (e_{i_1}, \dots, e_{i_k})
  \end{displaymath}
  \begin{prop}
    The wedge products
    $\{
    \alpha^I = \alpha^{i_1} \wedge \dots \wedge \alpha^{i_k}
    \cond 
    1 \le i_1 < \dots < i_k \le n
    \}
    $
    form a basis for $A_k(V)$.
  \end{prop}
  We will prove this very soon.
\end{frame}
\begin{frame}
  \begin{lem}
    Given
    $I = (i_1, \dots, i_k)$ with 
    $
    1 \le i_1 < \dots < i_k \le n
    $
    and 
    $J = (j_1, \dots, j_k)$ with 
    $
    1 \le j_1 < \dots < j_k \le n
    $
    we have
    \begin{displaymath}
      \alpha^I(e_J) = 
      \begin{cases}
        1 & \text{if $I=J$} \\
        0 & \text{if $I\ne J$}
      \end{cases}
    \end{displaymath}
  \end{lem}
  \begin{proof}
    We know that 
    \begin{displaymath}
      \alpha^I(e_J) = \alpha^{i_1} \wedge \dots \wedge \alpha^{i_k}
      (e_{j_1}, \dots, e_{j_k}) = \det [\alpha^{i_r}(e_{j_s})]
    \end{displaymath}
    For $I = J$, this is an identity matrix, so $\det [\alpha^{i_r}(e_{j_s})] =
    1$.
    For $I \ne J$ we can find an $r$ such that $i_r$ is not equal to any $j_s$.
    Thus the matrix has a zero row, and the determinant is zero.
  \end{proof}
\end{frame}

\begin{frame}
  \begin{prop}
    The wedge products
    $\{
    \alpha^I = \alpha^{i_1} \wedge \dots \wedge \alpha^{i_k}
    \cond 
    1 \le i_1 < \dots < i_k \le n
    \}
    $
    form a basis for $A_k(V)$.
  \end{prop}
  \begin{proof}
    Linearly independent: suppose $\sum_I c_I \alpha^I = 0$,
    $c_I \in \R$. Evaluate at any $e_J$:
    \begin{displaymath}
      0 = \sum_I c_I \alpha^I(e_J) = c_J
    \end{displaymath}

    Generators: Given $f \in A_k(V)$, let $c_I = f(e_I)$.

    Claim: $f = \sum_I c_I \alpha^I$.

    Since both sides are alternating it suffices to show that 
    evaluating $f$ and $\sum_I c_I \alpha^I$ 
    at each $e_J$ with $J = (j_1 < \dots < j_k)$
    gives the same result.
    
    By the lemma, 
    \begin{displaymath}
      \sum_{I} c_I \alpha^I(e_J) = c_J = f(e_J)
    \end{displaymath}
  \end{proof}
\end{frame}

\begin{frame}
  \begin{prop}
    The wedge products
    $\{
    \alpha^I = \alpha^{i_1} \wedge \dots \wedge \alpha^{i_k}
    \cond 
    1 \le i_1 < \dots < i_k \le n
    \}
    $
    form a basis for $A_k(V)$.
  \end{prop}
  \begin{cor}
    Suppose $\dim V = n$.
    \begin{enumerate}
      \item if $0 \le k \le n$, then $\dim A_kV = \binom n k$
      \item if $k > n$, then $A_kV = 0$.
    \end{enumerate}
  \end{cor}
\end{frame}
